\documentclass[dvisvgm,hypertex,aspectratio=169]{beamer}
\usefonttheme{serif}

\usepackage{animate}
\usepackage{ifthen}


%%%%%%%%%%%%%%%%%%%%%%%%%%%%%%%%%%%%%%%%%%%%%%%%%%%%%%%%%%%%%%%%%%%%%%%%%%%%%%%
% PageDown, PageUp key event handling; navigation symbols
%%%%%%%%%%%%%%%%%%%%%%%%%%%%%%%%%%%%%%%%%%%%%%%%%%%%%%%%%%%%%%%%%%%%%%%%%%%%%%%
% \usepackage[totpages]{zref}
% \usepackage{atbegshi}
% \usepackage{fontawesome}
% \setbeamertemplate{navigation symbols}{}
% \AtBeginShipout{%
%   \AtBeginShipoutAddToBox{%
%     \special{dvisvgm:raw
%       <defs>
%       <script type="text/javascript">
%       <![CDATA[
%         document.addEventListener('keydown', function(e){
%           if(e.key=='PageDown'){
%             \ifnum\thepage<\ztotpages
%               document.location.replace('\jobname-\the\numexpr\thepage+1\relax.svg');%
%             \fi
%           }else if(e.key=='PageUp'){
%             \ifnum\thepage>1
%             %document.location.replace('\jobname-\the\numexpr\thepage-1\relax.svg');%
%               document.location.replace('\jobname-\makeatletter\@anim@pad{2}{\thepage-1}\makeatother\relax.svg');%
%             \fi%
%           }
%         });
%       ]]>
%       </script>
%       </defs>
%     }%
%   }%
%   \AtBeginShipoutUpperLeftForeground{%
%     \raisebox{-\dimexpr\height+0.5ex\relax}[0pt][0pt]{\makebox[\paperwidth][r]{%
%       \normalsize\color{structure!40!}%
%       \ifnum\thepage>1%
%       \href{\jobname-\the\numexpr\thepage-1\relax.svg}{\faArrowLeft}%
%       \else%  
%         \textcolor{lightgray}{\faArrowLeft}%  
%       \fi\hspace{0.5ex}%
%       \ifnum\thepage<\ztotpages%
%       \href{\jobname-\the\numexpr\thepage+1\relax.svg}{\faArrowRight}%
%       \else%
%         \textcolor{lightgray}{\faArrowRight}%  
%       \fi%
%       \hspace{0.5ex}%
%     }}%
%   }%  
% }%
% %%%%%%%%%%%%%%%%%%%%%%%%%%%%%%%%%%%%%%%%%%%%%%%%%%%%%%%%%%%%%%%%%%%%%%%%%%%%%%%

\usepackage{animate}
\usepackage{tikz}
\usepackage{circuitikz}
%\usetikzlibrary{external}
%\tikzexternalize[prefix=tikzfigures/] % activate and define figures/ as cache folder

% ------------------------------------------------
% Determine which slides to include

\includeonlyframes{%
P0,% Phasors
P1A,% Phasor animation
}
% ------------------------------------------------


% ------------------------------------------------
% Dimensions

\def\axissize{6}
% ------------------------------------------------


% pu values
% \phasediagram{iq}{id}{E}{vcolor}{icolor}{fcolor}
\newcommand{\phasordiagram}[6]{%
  % Magnet
  \shade[left color=green!90, right color = red!90, opacity = 0.7]  (-0.3, -0.1) rectangle (0.3, 0.1); 
  \def\vscale{2}
  \def\iscale{1}
  \def\mscale{0.7}
  \pgfmathsetmacro{\Xr}{0.3*#3} % Reactance
   \pgfmathsetmacro{\ee}{\vscale*#3}
    \draw[blue!80,->] (0,0) -- node[right, pos=0.6]{$E=#3$} (0,\ee);
    \pgfmathsetmacro{\iq}{\iscale*#1}
    \pgfmathsetmacro{\id}{-\iscale*#2}
    \pgfmathsetmacro{\idtest}{100**#2}
    
    \draw[#5, thin, ->] (0,0) -- node[above, near end] {$I_d$} (\id, 0);
    \pgfmathsetmacro{\iimag}{sqrt(#2*#2 + #1*#1)}

    \pgfmathsetmacro{\ixmag}{\Xr*\iimag}
    \draw[#5, thin, ->] (\id,0) --  node[left, near end] {$I_q$} (\id,\iq);
    \draw[#5, ->] (0,0) -- node[right, near end] {$I=\iimag$} (\id, \iq);
    \pgfmathsetmacro{\ixx}{\vscale*\Xr*#1}
    \pgfmathsetmacro{\ixy}{\vscale*\Xr*#2}
        \pgfmathsetmacro{\vvmag}{sqrt((\Xr*#1)*(\Xr*#1) + (#3-\Xr*#2)*(#3-\Xr*#2))}
    \draw[#4,->] (0,\ee) --  node[above] {$I(#3X)=\ixmag$} ++(-\ixx,-\ixy);
    \draw[#4,<-] (0,\ee) ++(-\ixx,-\ixy) -- node[left] {$V$} (0,0);
    \pgfmathsetmacro{\phimag}{2}
    \draw[green!80!black, dashed, ->] (0,0) -- node[above] {$\phi_{mag}$} (\phimag,0);
    \draw[#6, dashed, ->] (\phimag,0) -- node[right] {$\phi_{arm}$} (\phimag+\mscale*\id,\mscale*\iq);
    \draw[#6, dashed, ->] (0,0) -- node[right] {$\phi_{res}$} (\phimag+\mscale*\id,\mscale*\iq);

    \draw[dotted,] (0, 2.08) arc [start angle=90, end angle=130, radius = 2.08] node[above] {$V_r$};
    \draw[blue!80,->] (0,0) -- node[right, pos=0.6]{$E=#3$} (0,\ee);
  }


% First argument is time, second is pole number, third is amplitude,
% fourth is phase shift, fifth is color
\newcommand*{\drawwave}[5]{%
  \pgfmathsetmacro{\axl}{12}
  \draw[#5, domain=0:360, samples=36, smooth, variable=\t]
  plot ( { \t/360*\axl }, {-#3*(sin(\n)*cos(\t*#2/2 - 90 -#4) } );
}


\author{Kjartan Halvorsen}
\date{2021-06-01}
\title{Field-oriented control of permanent magnet synchronous motors}



\begin{document}

%---------------------------------------------------------
\begin{frame}[label=P1A]{Phasors}

  \def\vcolor{blue!80!black}
  \def\icolor{green!80!black}

  \pgfmathsetmacro{\Gcmag}{1.5}
  \def\phshift{\frac{\pi}{3}}
  \pgfmathsetmacro{\thetaval}{(60)}
  \pgfmathsetmacro{\vrms}{1.2}
  \pgfmathsetmacro{\vmag}{sqrt(2)*\vrms}
  \pgfmathsetmacro{\irms}{\Gcmag*\vrms}
  \pgfmathsetmacro{\iimag}{sqrt(2)*\irms}
  \pgfmathsetmacro{\axl}{4}
  \pgfmathsetmacro{\ws}{12}

  \[ \textcolor{\vcolor}{r(t) = \sin(\omega_1 t)}, \qquad \textcolor{\icolor}{y(t) = \Gcmag \sin(\omega_1 t - \phshift )} \] 

 \begin{center}
    \begin{animateinline}[controls,autoplay,loop]{8}
      \multiframe{120}{n=0+3}{
        \begin{tikzpicture}[scale=0.6, transform shape]
          %\useasboundingbox (-4 cm, -4 cm) rectangle (14 cm, 8 cm);
          \pgfmathsetmacro{\vcy}{\vmag*sin(\n)}
          \pgfmathsetmacro{\icy}{\iimag*sin(\n-\thetaval)}
           \begin{scope}[xshift = 2.8cm,]
             \draw[->, black!50] (0 cm, 0) -- (2*\axl cm, 0);
             %\foreach \ax in {0, 60, 120, ..., 360} {
             % \pgfmathsetmacro{\axx}{\ax*\axl/360.0)}
             % \draw (\axx cm, 0) -- (\axx cm, -0.5) node[below] {\ax};
             %}
             \pgfmathsetmacro{\dom}{\n + 360}
             \pgfmathsetmacro{\nnow}{\dom/360*\axl}
             \draw[black!30] (\nnow, -3 cm) -- (\nnow cm, 3cm);
             %\node[black!30] at (2*\axl, -0.5) {$t$};
             \node[\vcolor, circle, draw, inner sep=1pt] at (\nnow, \vcy) {};
             \node[\icolor, circle, draw, inner sep=1pt] at (\nnow, \icy) {};
             \draw[\vcolor, domain=0:\dom, smooth, variable=\t]
             plot ( { \t/360.0*\axl}, {\vmag*sin(\t)} );
             %plot ( { \t/360.0*\axl}, {\vmag*sin(\dom -\t)} );
             \draw[\icolor, domain=0:\dom, smooth, variable=\t]
             plot ( { \t/360.0*\axl}, {\iimag*sin(\t-\thetaval)} );
             %plot ( { \t/360.0*\axl}, {\iimag*sin(\dom -\t-\thetaval)} );
           \end{scope}

          \begin{scope}[xshift=-1cm,]

            \draw[->, thin, black!50] (-3, 0) -- (3, 0);
            \draw[->, thin, black!50] (0,-3) -- (0, 3);
            %\node[\vcolor, ] at (\vmag, -0.5)  {$r$}; 
            %\node[\icolor] at (\irms, -1.6) {$y$}; 

            \pgfmathsetmacro{\iiy}{\iimag*sin(-\thetaval)}
            \pgfmathsetmacro{\iix}{\iimag*cos(\thetaval)}
            \draw[\vcolor,thick, ->] (0,0) -- (\vmag, 0) node[right] {$r$};
            \draw[\icolor,thick, ->] (0,0) -- (\iix, \iiy) node[right] {$y$};

            \pgfmathsetmacro{\vcx}{\vmag*cos(\n)}
            \draw[\vcolor,  ->] (0,0) -- (\vcx, \vcy); 

            \pgfmathsetmacro{\icx}{\iimag*cos(\n-\thetaval)}
            \draw[\icolor,  ->] (0,0) -- (\icx, \icy); 
          \end{scope}
        \end{tikzpicture}    
      }
    \end{animateinline}
  \end{center}
\end{frame}

\note{%
  Understanding the field-oriented control, requires to understand phasors.

  A phasor in the context of AC theory is a vector that represents sinusoidally varying quantities such as voltage and current. The magnitude represents the RMS value of the sinusoid, and the angle of the phasor represents the phase of the sinusoid. In this case we have a voltage phasor and a current phasor, and we see that the voltage phasor leads the curent phasor with a certain angle \theta. If these represents the voltage and current in a simple circuit, then the impedance of that circuit is both resistive and reactive (inductive) with zero resistance, the angle would be 90.  
}

\end{document}

