% Created 2022-10-07 Fri 07:47
% Intended LaTeX compiler: pdflatex
\documentclass[letterpaper]{scrartcl}
\usepackage[utf8]{inputenc}
\usepackage[T1]{fontenc}
\usepackage{graphicx}
\usepackage{grffile}
\usepackage{longtable}
\usepackage{wrapfig}
\usepackage{rotating}
\usepackage[normalem]{ulem}
\usepackage{amsmath}
\usepackage{textcomp}
\usepackage{amssymb}
\usepackage{capt-of}
\usepackage{hyperref}
\usepackage{khpreamble}
\usepackage[top=10mm, bottom=10mm]{geometry}
\author{Kjartan Halvorsen}
\date{}
\title{Characterization of a DC motor}
\hypersetup{
 pdfauthor={Kjartan Halvorsen},
 pdftitle={Characterization of a DC motor},
 pdfkeywords={},
 pdfsubject={},
 pdfcreator={Emacs 26.3 (Org mode 9.4.6)}, 
 pdflang={English}}
\begin{document}

\maketitle

\section*{Modeling}
\label{sec:org9eca945}

\subsection*{ODE}
\label{sec:org2b1d6d7}
The two coupled differential equations that govern the behavior of the DC motor with constant stator magnetic field are
\begin{align}
L \frac{d}{dt} i(t) &= u(t) - Ri(t) - kg_r\omega(t), \label{eq:circ}\\
J \frac{d}{dt} \omega(t) &= kg_r i(t) - T_l(t)
\label{eq:newton}
\end{align}
with variables
\begin{description}
\item[{\(i(t)\)}] The armature current,
\item[{\(\omega(t)\)}] The angular velocity of the wheel,
\item[{\(u(t)\)}] The armature voltage,
\item[{\(T_l(t)\)}] The load torque,
\end{description}
and parameters
\begin{description}
\item[{\(L\)}] The armature inductance,
\item[{\(R\)}] The armature resistance,
\item[{\(k\)}] The motor constant (in Nm/A or V/(rad/s)),
\item[{\(g_r\)}] The gear ratio
\item[{\(J\)}] The moment of inertia.
\end{description}

\subsection*{Transfer functions}
\label{sec:org51b2c71}
Taking the Laplace transform of \eqref{eq:circ} and \eqref{eq:newton} (and ignoring the initial values) gives
\begin{align}
sL I(s) &= U(s) - RI(s) - kg_r\Omega(s), \\
\label{eq:circL}
sJ\Omega(s) &= kg_r I(s) - T_l(s).
\label{eq:newtonL}
\end{align}
Solving \eqref{eq:circL} for \(I(s)\) gives
\[ I(s) = \frac{1}{sL+R}U(s) - \frac{kg_r}{sL + R} \]
and inserting in \eqref{eq:newtonL} gives
\begin{equation}
\begin{aligned}
sJ\omega(s) &= \frac{k g_r}{sL+R}U(s) - \frac{(kg_r)^2}{sL+R}\Omega(s) - T_l(s)\\
sJ(sL+R)\Omega(s) &= k g_rU(s) - (kg_r)^2\Omega(s) - (sL+R)T_l(s)\\
\big(sJ(sL+R) + (k g_r)^2\big)\Omega(s) &= k g_rU(s) - (sL+R)T_l(s)\\
\Omega(s) &= \frac{k g_r}{sJ(sL+R) + (k g_r)^2}U(s) - \frac{sL+R}{sJ(sL+R) + (k g_r)^2}T_l(s)
\end{aligned}
\end{equation}
\end{document}