% Created 2021-08-30 Mon 12:43
% Intended LaTeX compiler: pdflatex
\documentclass[letterpaper]{scrartcl}
\usepackage[utf8]{inputenc}
\usepackage[T1]{fontenc}
\usepackage{graphicx}
\usepackage{grffile}
\usepackage{longtable}
\usepackage{wrapfig}
\usepackage{rotating}
\usepackage[normalem]{ulem}
\usepackage{amsmath}
\usepackage{textcomp}
\usepackage{amssymb}
\usepackage{capt-of}
\usepackage{hyperref}
\usepackage{khpreamble}
\usepackage[top=10mm, bottom=10mm]{geometry}
\author{Kjartan Halvorsen}
\date{}
\title{Pole placement, root locus and lead-lag}
\hypersetup{
 pdfauthor={Kjartan Halvorsen},
 pdftitle={Pole placement, root locus and lead-lag},
 pdfkeywords={},
 pdfsubject={},
 pdfcreator={Emacs 26.3 (Org mode 9.4.6)}, 
 pdflang={English}}
\begin{document}

\maketitle

\section*{The inverted pendulum}
\label{sec:orga27804c}
The friction-less, inverted pendulum has two poles symmetric about the imaginary axis.

\subsection*{PD-control}
\label{sec:org2447a44}
\def\omegazero{1}
     \begin{center}
       \small
       \begin{tikzpicture}[scale = 0.8, node distance=20mm, block/.style={rectangle, draw, minimum width=12mm}, sumnode/.style={circle, draw, inner sep=2pt}]
       
       \node[coordinate] (refinput) {};
       \node[sumnode, right of=refinput, node distance=20mm] (sumerr) {\tiny $\sum$};
       \node[block, right of=sumerr] (controller) {$K(2s + a)$};
       \node[above of=controller, node distance=6mm] {controller};
       \node[block, right of=controller, node distance=24mm] (plant) {$\frac{1}{s^2 - a^2}$};
       \node[above of=plant, node distance=6mm] {plant};
       \node[coordinate, right of=plant, node distance=20mm] (output) {};

       \draw[->] (refinput) -- node[above, pos=0.3] {$y_{ref}(t)$} (sumerr);
       \draw[->] (sumerr) -- node[above] {$e(t)$} (controller);
       \draw[->] (controller) -- node[above] {$u(t)$} (plant);
       \draw[->] (plant) -- node[coordinate] (measure) {} node[above, pos=0.8] {$y(t)$} (output);
       \draw[->] (measure) -- ++(0,-14mm) -| node[right, pos=0.95] {$-$} (sumerr);

    \begin{axis} [
        xshift = 12cm,
        yshift = -3cm,
        width=12cm,
        height=8cm,
        axis lines=middle,
        axis line style={->},
        xtick={-1, 1},
        ytick={-1, 1},
        xticklabels={$-a$, $a$},
        xmin=-6,
        xmax=2,
        ymin=-3,
        ymax=3,
        ytick=\empty,
        xlabel=Re,
        ylabel=Im,
        ]
        
        \addplot [ thick,black, mark=x, mark size=6pt, only marks] coordinates { (-\omegazero,0) (\omegazero,0) }; 
        %\addplot [ thick,black, mark=o, mark size=6pt] coordinates { (-0.5,0) }; 
        
      %\node[coordinate, pin={[pin distance=3cm] 135:{3 startpunkter}}] at (axis cs:0,0) {};
      %\node[coordinate, pin={[pin distance=2.5cm] -135:{2 ändpunkter}}] at (axis cs:-0.5,0) {};
    \end{axis}

       \end{tikzpicture}


      
     \end{center}

\subsection*{Modified PD-control}
\label{sec:orgdc47f6d}
 \def\omegazero{1}
 \begin{center}
   \small
   \begin{tikzpicture}[scale = 0.8, node distance=20mm, block/.style={rectangle, draw, minimum width=12mm}, sumnode/.style={circle, draw, inner sep=2pt}]

   \node[coordinate] (refinput) {};
   \node[sumnode, right of=refinput, node distance=20mm] (sumerr) {\tiny $\sum$};
   \node[block, right of=sumerr] (controller) {$K\frac{s + 0.5a}{s+5a}$};
   \node[above of=controller, node distance=6mm] {controller};
   \node[block, right of=controller, node distance=24mm] (plant) {$\frac{1}{s^2 - a^2}$};
   \node[above of=plant, node distance=6mm] {plant};
   \node[coordinate, right of=plant, node distance=20mm] (output) {};

   \draw[->] (refinput) -- node[above, pos=0.3] {$y_{ref}(t)$} (sumerr);
   \draw[->] (sumerr) -- node[above] {$e(t)$} (controller);
   \draw[->] (controller) -- node[above] {$u(t)$} (plant);
   \draw[->] (plant) -- node[coordinate] (measure) {} node[above, pos=0.8] {$y(t)$} (output);
   \draw[->] (measure) -- ++(0,-14mm) -| node[right, pos=0.95] {$-$} (sumerr);

\begin{axis} [
    xshift = 12cm,
    yshift = -3cm,
    width=12cm,
    height=8cm,
    axis lines=middle,
    axis line style={->},
    xtick={-1, 1},
    ytick={-1, 1},
    xticklabels={$-a$, $a$},
    xmin=-7,
    xmax=2,
    ymin=-3,
    ymax=3,
    ytick=\empty,
    xlabel=Re,
    ylabel=Im,
    ]

    \addplot [ thick,black, mark=x, mark size=6pt, only marks] coordinates { (-\omegazero,0) (\omegazero,0) }; 
    %\addplot [ thick,black, mark=o, mark size=6pt] coordinates { (-0.5,0) }; 

  %\node[coordinate, pin={[pin distance=3cm] 135:{3 startpunkter}}] at (axis cs:0,0) {};
  %\node[coordinate, pin={[pin distance=2.5cm] -135:{2 ändpunkter}}] at (axis cs:-0.5,0) {};
\end{axis}

   \end{tikzpicture}



 \end{center}





\section*{The DC-motor}
\label{sec:org5250fe6}
The dynamics of the velocity of the DC motor is a first-order system when the input is armature voltage. By integrating once, we get the shaft angle as the output.

\subsection*{PD-control}
\label{sec:org5073a75}
\def\omegazero{1}
     \begin{center}
       \small
       \begin{tikzpicture}[scale = 0.6, node distance=20mm, block/.style={rectangle, draw, minimum width=12mm}, sumnode/.style={circle, draw, inner sep=2pt}]
       
       \node[coordinate] (refinput) {};
       \node[sumnode, right of=refinput, node distance=20mm] (sumerr) {\tiny $\sum$};
       \node[block, right of=sumerr] (controller) {$K(s + 2a)$};
       \node[above of=controller, node distance=6mm] {controller};
       \node[block, right of=controller, node distance=24mm] (plant) {$\frac{k}{s+a}$};
       \node[block, right of=plant, node distance=20mm] (int) {$\frac{1}{s}$};
       %\node[above of=plant, node distance=6mm] {plant};
       \node[coordinate, right of=int, node distance=20mm] (output) {};

       \draw[->] (refinput) -- node[above, pos=0.3] {$y_{ref}(t)$} (sumerr);
       \draw[->] (sumerr) -- node[above] {$e(t)$} (controller);
       \draw[->] (controller) -- node[above] {$u(t)$} (plant);
       \draw[->] (plant) -- node[above] {} (int);
       \draw[->] (int) -- node[coordinate] (measure) {} node[above, pos=0.8] {$y(t)$} (output);
       \draw[->] (measure) -- ++(0,-14mm) -| node[right, pos=0.95] {$-$} (sumerr);

    \begin{axis} [
        xshift = 18.5cm,
        yshift = -3cm,
        width=12cm,
        height=8cm,
        axis lines=middle,
        axis line style={->},
        xtick={-1, 0},
        ytick={-1, 0},
        xticklabels={$-a$, $0$},
        xmin=-6,
        xmax=2,
        ymin=-3,
        ymax=3,
        ytick=\empty,
        xlabel=Re,
        ylabel=Im,
        ]
        
        \addplot [ thick,black, mark=x, mark size=6pt, only marks] coordinates { (-\omegazero,0) (0,0) }; 
        %\addplot [ thick,black, mark=o, mark size=6pt] coordinates { (-0.5,0) }; 
        
      %\node[coordinate, pin={[pin distance=3cm] 135:{3 startpunkter}}] at (axis cs:0,0) {};
      %\node[coordinate, pin={[pin distance=2.5cm] -135:{2 ändpunkter}}] at (axis cs:-0.5,0) {};
    \end{axis}

       \end{tikzpicture}


      
     \end{center}

\subsection*{Lag-compensator}
\label{sec:orgef421f0}
A lag compensator is a controller with one pole and one zero, and where the pole is closer to the origin than the zero. Such a compensator increases the gain at low frequencies, at the expense of more oscillatory response.

 \def\omegazero{2}
 \begin{center}
   \small
   \begin{tikzpicture}[scale = 0.6, node distance=20mm, block/.style={rectangle, draw, minimum width=12mm}, sumnode/.style={circle, draw, inner sep=2pt}]

   \node[coordinate] (refinput) {};
   \node[sumnode, right of=refinput, node distance=20mm] (sumerr) {\tiny $\sum$};
   \node[block, right of=sumerr] (controller) {$K\frac{s + a}{4s+a}$};
   \node[above of=controller, node distance=6mm] {controller};
   \node[above of=controller, node distance=6mm] {controller};
   \node[block, right of=controller, node distance=24mm] (plant) {$\frac{k}{s+a}$};
   \node[block, right of=plant, node distance=20mm] (int) {$\frac{1}{s}$};
   %\node[above of=plant, node distance=6mm] {plant};
   \node[coordinate, right of=int, node distance=20mm] (output) {};

   \draw[->] (refinput) -- node[above, pos=0.3] {$y_{ref}(t)$} (sumerr);
   \draw[->] (sumerr) -- node[above] {$e(t)$} (controller);
   \draw[->] (controller) -- node[above] {$u(t)$} (plant);
   \draw[->] (plant) -- node[above] {} (int);
   \draw[->] (int) -- node[coordinate] (measure) {} node[above, pos=0.8] {$y(t)$} (output);
   \draw[->] (measure) -- ++(0,-14mm) -| node[right, pos=0.95] {$-$} (sumerr);

\begin{axis} [
    xshift = 18.5cm,
    yshift = -3cm,
    width=12cm,
    height=8cm,
    axis lines=middle,
    axis line style={->},
    xtick={-1, 0},
    ytick={-1, 0},
    xticklabels={$-a$, $0$},
    xmin=-6,
    xmax=2,
    ymin=-3,
    ymax=3,
    ytick=\empty,
    xlabel=Re,
    ylabel=Im,
    ]

    \addplot [ thick,black, mark=x, mark size=6pt, only marks] coordinates { (-\omegazero,0) (0,0) }; 
    %\addplot [ thick,black, mark=o, mark size=6pt] coordinates { (-0.5,0) }; 

  %\node[coordinate, pin={[pin distance=3cm] 135:{3 startpunkter}}] at (axis cs:0,0) {};
  %\node[coordinate, pin={[pin distance=2.5cm] -135:{2 ändpunkter}}] at (axis cs:-0.5,0) {};
\end{axis}

   \end{tikzpicture}


 \end{center}
\end{document}